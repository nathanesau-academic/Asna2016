\documentclass[12pt]{article}

\usepackage[margin=1.0in]{geometry}
\usepackage{hyperref}

\hypersetup{
	colorlinks,
	linkcolor={blue},
	linktoc=page,
	urlcolor=blue
}

\setcounter{secnumdepth}{0}

\begin{document}

\section{Case Competition Final Round}

\subsection{Background}

Based on the feedback GoStats has given to them, administrators of Ontario University have begun voicing concerns regarding the potential volatility of self-insuring (particular in regards to potential bodily injury liability claims). As such, they have asked that you investigate third part insurance options  (i.e. purchasing fleet insurance for the automobiles). You will be expected to give a detailed presentation that makes recommendations for the actions OU should take in regards to their new auto fleet as a whole (whether or not to self-insure, and what safety features to purchase).

\subsection{Task}

Using information from the first round of the competition, you are tasked with finding the net break-even premium of purchasing third party insurance (i.e. breakeven in terms of only cost to the third party insurance company, for the cases with and without safety features). Determining these prices will help OU benchmark the quotations it may receive for fleet insurance to help it make a more informed decision. After this, you should evaluate, at those prices, what actions OU should take.

At a minimum, you will need to conduct a thorough cost-benefit analysis between self-insuring and purchasing insurance, and examining how either decision will impact OU as a whole.

\subsection{Assumptions}

You are expected to make assumptions when necessary, since not all of the data is available. Please list your assumptions and explain how you made them and why they would be reasonable assumptions to make (of if you obtained them from your research, where they were found).

Some questions that you will want to consider:

\begin{itemize}
\item What is the breakeven insurance premium without your safety features?
\item What is the breakeven insurance premium with your safety features?
\item Have you communicated to the audience the steps you took, and the graphs of your results?
\item What is the minimum discount that would make safety features worthwhile in the long run?
\item What are the (holistic and financial) impacts to OU if they were to no longer able to self-insurance?
\end{itemize}

\subsection{Scoring}

Judges will be considering the following metrics when assessing your presentations:

\begin{itemize}
\item The quality of the solution provided
\item The quality of the presentation of the solution
\end{itemize}

It will be more than likely be helpful to quickly summarize your findings from the first round, so that the audience will better understand the optimal safety configuration you have chosen. Doing so will likely improve your presentation overall.

\subsection{Submissions}

This final round of the case competition will be presented live at ASNA 2016 Niagara Falls. Your team will be assigned a timeslot to present your findings that will be communication to you upon receiving your submissions. All presentations must be in the form of a Powerpoint and set to \href{mailto:casecompetition@anea-asna.ca}{\texttt{casecompetition@anea-asna.ca}} by January 8th, 12:01am ET.

\end{document}

