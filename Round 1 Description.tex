\documentclass[12pt]{article}

\usepackage[utf8]{inputenc}
\usepackage{hyperref}

\usepackage{url}

\expandafter\def\expandafter\UrlBreaks\expandafter{\UrlBreaks%  save the current one
  \do\a\do\b\do\c\do\d\do\e\do\f\do\g\do\h\do\i\do\j%
  \do\k\do\l\do\m\do\n\do\o\do\p\do\q\do\r\do\s\do\t%
  \do\u\do\v\do\w\do\x\do\y\do\z\do\A\do\B\do\C\do\D%
  \do\E\do\F\do\G\do\H\do\I\do\J\do\K\do\L\do\M\do\N%
  \do\O\do\P\do\Q\do\R\do\S\do\T\do\U\do\V\do\W\do\X%
  \do\Y\do\Z}

\usepackage[margin=1.0in]{geometry}
\setcounter{secnumdepth}{0}
\usepackage{color}

\hypersetup{
	colorlinks,
	linkcolor={blue},
	linktoc=page,
	urlcolor=blue
}

\begin{document}

\section{Ontario University: Actuarial Society Case Study}

\subsection{Background}

Ontario University (OU) is one of the premier universities in the world. Over the years, OU has come under increasing financial pressure due to diminishing funding from the province of Ontario. Over the past 10 years funding decreased 25\%, while OU enrollment grew 50\%. In order to respond to these financial pressures while maintaining its ability to offer world-class education, OU has instituted as series of initiatives to increase revenue (from donations, grants, tuition and investments) and decrease expenses.

\subsection{Case Study}

Your actuarial firm, GoStats, provides property/casualty actuarial consulting services to OU. OU has a fleet of 950 aging autos, and a major donor (Frederica Firestone) just donated OU money to replace its entire fleet. Unfortunately, the donation is only enough to cover the purchase of standard model vehicles. OU is aware that over the past several years there has been a lot of innovation related to auto safety, and these innovative safety features are not included in standard model vehicles. OU has asked GoStats to investigate whether the new cars they purchase should include some of these safety features. The people who will be evaluating this decision include representatives from the following operations:

\begin{enumerate}
\item \textbf{Risk}: Responsible for student, visitor and employee safety as well as managing the cost of auto liability, auto physical damage, and other insurable risks. They are particularly interested in potential auto cost savings. In order for your recommendations to move forward they need to be convinced of the return on investment (ROI) of these safety features, and they need to understand the payback period.
\item \textbf{Operations}: These represent the areas of OU that will purchase and use the vehicles. While the Risk Department would pay to cover the additional cost of these new safety features, the Operations Department would need to pay for any driver training or increased repair costs resulting from the safety features.
\item \textbf{Finance}: Responsible for the overall finances of OU. They are involved due to the size of this potential safety investment, and they will need to be convinced that the ROI is higher than what they could get if the money were otherwise invested (for example in conservative financial instruments)
\item \textbf{Marketing/ Public Relations}: Marketing is interested to know if this safety initiative could further enhance OU's renowned reputation as an innovative leader among universities that treats its employees and students with the upmost regard.
\end{enumerate}

\medskip\noindent
OU is self-insured for auto exposures through a special arrangement with the Financial Services Commission of Ontario (FSCO). This means that rather than purchasing coverage from an insurance company, OU directly pays for costs associated with auto accidents involving OU vehicles. Costs related to auto accidents are broken down into the following categories:
%
\begin{itemize}
\item Auto physical damage: generally the cost of replacing or repairing the OU vehicle
\item Auto property damage liability: the cost associated with property (other than the OU car) that is damaged in an OU car accident. For example if an OU car hits another car, the OU driver is at fault, and if other other car is damaged, then costs associated with repairing/ replacing the other car falls under this category
\item Auto bodily injury: the cost associated with people who are injured by an OU driver in a car accident. For example, if a OU car hits another car, the OU driver is at fault, and if the driver of the other car is injured, then costs associated with the other driver falls under this category
\item Auto accident benefits: costs associated with medical payments made to an OU driver resulting from an injury in an OU car accident
\end{itemize}

\medskip\noindent While it's good to understand the categories of auto costs that were just outlined, your estimate of costs and savings can lump all these categories together.

You will be expected to give a presentation that makes recommendations regarding what safety features OU should be purchasing for the new fleet. You should address what you think will be the key concerns of the evaluators from each department. There may be some issues that you are not able to quantify to your satisfaction. That's okay. Make an educated guess, and state any additional assumptions you needed to make. Bonus points if your incorporate additional secondary advantages or disadvantages to OU regarding the safety features (for example employee/ student healthcare). You are expected to be factual and unbiased; you have no investment in whether or not OU purchases any of the auto safety features.

At a minimum you will need to forecast ultimate costs associated with auto claims under two scenarios, one in which OU dooes not purchase any of the potential safety features and one in which they purchase the features you are recommending. If you recommend not purchasing any of the potential safety features, then you should explain why. You do not need to be concerned with the basic cost of each new car, only the incremental cost associated with any safety features you are recommending.

At a minimum, you should consider the following safety features. Through your research you may come across additional safety features, and you may include these in your presentation if you want.
%
\begin{itemize}
\item Computerized collision avoidance
\item Passenger-side airbags
\item Adaptive headlights
\item Lane departure warning
\item Parking assistant technology (ultrasonic parking sensors)
\end{itemize}

\subsection{Assumptions}

\begin{itemize}
\item OU's fleet will be made up of mid-size private passenger sedans
\item The OU auto fleet will be replaced next year. The average life of a new car is 10 years.
\end{itemize}

\subsection{Data}

\begin{itemize}
\item You have been provided with an Excel workbook that will be useful in developing a forecast of OU auto costs. The data in this workbook is based on OU's current fleet, which does not have any of the safety features you are considering.
\end{itemize}

\subsection{Default Optional Safety Features}

\begin{itemize}
\item Safety feature A: Computerized collision avoidance package
\begin{itemize}
\item Cost: \$2,000 one-time cost per vehicle, \$100 maintenance fee per vehicle per year
\item Effectiveness: 50\% reduction in accident frequency for all categories of auto combined
\end{itemize}
\item Safety feature B: Passenger-side airbag installation
\begin{itemize}
\item Cost: \$1,500 one-time cost per vehicle
\item Effectiveness: 20\% reduction in accident severity for all categories of auto combined
\end{itemize}
\item Note that some expenses (such as staff salaries) may not be reduced by the same proportion
\item You are encouraged to do research on more specific safety features to replace the default options in your case study using the suggested resources below
\item You are also encouraged to discuss non-financial impacts of your chosen decision
\end{itemize}

\subsection{Suggested resources}

\begin{itemize}
\item Cost of safety features: these may be quantified at various car-pricing websites such as \url{cars.com}, \url{kbb.com} and varios car manufacturer websites
\item Safety research
\begin{itemize}
\item \url{http://www.iihs.org/iihs/topics/t/crash-avoidance-technologies/presentations/bytag} 
\item \url{http://www.iihs.org/iihs/topics/t/crash-avoidance-technologies/hldi-research}
\item Highway Loss Data Institue: Bulletin Vol. 31, No. 16 (September 2014): Honda Accord collision avoidance features: an update
\item Highway Loss Data Institute: Bulletin Vol. 28, No. 22 (December 2011): Buick collision avoidance features: initial features 
\item Highway Loss Data Institute: TRB Annual Meeting (January 2014): Experiences of owners of non-luxury vehicles with collision avoidance technology
\item Highway Loss Data Institute: TRB Annual Meeting (January 2013): Measuring crash avoidance system effectiveness with insurance data
\end{itemize}
\item Actuarial methods ``Basic ratemaking'' \url{http://www.casact.org/library/studynotes/Werner_Modlin_Ratemaking.pdf}. Appendix A, pages 337 -- 358.
\item Actuarial Reserve Estimation \url{http://www.casact.org/library/studynotes/Friedland_estimating.pdf}. pp 57--60, 90--92, 392.
\end{itemize}

\section{Glossary of Terms and Definitions}

\begin{itemize}
\item \textbf{Allocated Loss Adjustment Expenses (ALAE)}: Claim-related expenses that are directly attributable to a specific claim; for example, fees associated with outside legal counsel hired to defend a claim can be directly assigned to a specific claim. 
\item \textbf{Auto Liability}: Insurance that protects the insured against financial loss because of legal liability for automobile-related injuries to others or damage to their property by an auto.
\item \textbf{Conservative Financial Instruments}: Examples include certificates of deposit, bonds, preferred stocks or mutual funds.
\item \textbf{Experience Period}: A period of history used by insurance companies to evaluate terms when issuing a policy. Any claim and/or premium amounts within this period will affect the current insurance premium rates the customer receives.
\item \textbf{Exposure}: An exposure is the basic unit of risk that underlies the insurance premium. The exposure measure used for ratemaking purposes varies considerably by line of business.  Some examples are number of vehicles, insured value of building, payroll, square footage, etc.  There are four different ways that insurers measure exposures: written, earned, unearned, and in-force exposures.
\item \textbf{Frequency}: The number of claim counts divided by the number of exposures.
\item \textbf{Incremental Cost}: The increase or decrease in costs as a result of one more or one less unit of output.
\item \textbf{Indemnity}: Compensation for (or to provide against) injury, loss, incurred penalties, or from a contingent liability.
\item \textbf{Inflation}: A sustained increase in the aggregate or general price level in an economy. Inflation means there is an increase in the cost of living.
\item \textbf{Loss Adjustment Expense (LAE)}: In addition to the money paid to the claimant for compensation, the insurer generally incurs expenses in the process of settling claims; these expenses are called loss adjustment expenses (LAE). Loss adjustment expenses can be separated into allocated loss adjustment expenses (ALAE) and unallocated loss adjustment expenses (ULAE): LAE = ALAE + ULAE
\item \textbf{Loss Costs}: Also called ``pure premium'', the actual or expected cost to an insurer of indemnity payments and allocated loss adjustment expenses (ALAEs) per exposure. Loss costs do not include overhead costs or profit loadings.
\item \textbf{On-level}: In this case study, on-level refers to inflation adjustments.  Claims costs in the historical periods are adjusted to today’s dollars.
\item \textbf{Premium}: The amount the insured pays for insurance coverage. The term can also be used to describe the aggregate amount a group of insureds pays over a period of time. Like exposures, there are written, earned, unearned, and in-force premium definitions.
\item \textbf{Pure Premium}: Refer to ``loss costs''
\item \textbf{Return on Investment (ROI)}: Measures the gain or loss generated on an investment relative to the amount of money invested. ROI is usually expressed as a percentage and is typically used for personal financial decisions, to compare a company's profitability or to compare the efficiency of different investments. 
\item \textbf{Self-insured}: The term used to describe a situation whereby a company opts to retain some of its potential financial risks, rather than to transfer those risks to a third party like an insurance company. In doing so the company chooses to pay its own losses arising from those risks.
\item \textbf{Severity}: The average claim amount determined as the total losses divided by the total claim counts.
\item \textbf{Ultimate Costs}: Estimate of total costs expected once all claims are settled and paid.
\item \textbf{Unallocated Loss Adjustment Expenses (ULAE)}: Claim-related expenses that cannot be directly assigned to a specific claim. For example, salaries of claims department personnel are not readily assignable to a specific claim and are categorized as ULAE. 
\item \textbf{Workers' Compensation}: Insurance that covers medical and rehabilitation costs and lost wages for employees injured at work; required by law in all states.
\end{itemize}

\end{document}
